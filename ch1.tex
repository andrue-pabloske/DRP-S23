\documentclass{article}

\usepackage{amsmath, amsthm}
\usepackage[margin=0.25in]{geometry}
\usepackage[mathcal]{euscript}
\usepackage[linesnumbered, lined, boxed, commentsnumbered]{algorithm2e}
\usepackage[noend]{algpseudocode}

\theoremstyle{definition}
\newtheorem{defn}{Definition}
%\AtEndEnvironment{defn}{\qed}

\theoremstyle{plain}
\newtheorem{theorem}{Theorem}

%\title{}
\title{Notes on Computational Geometry\\Chapter 1: Intro}
\author{Andy Pavlosky}
\date{}

\begin{document}

\maketitle

Computational geometry = study of algorithms and data structures for geometric objects.
Has applications in computer graphics, CAD, robotics, geographic information systems.

\section{Convex Hulls}

\begin{defn}
    A subset $S$ of the plane is called \textbf{convex} if for all points $p, q \in S$,
    the line segment $\overline{pq}$ is contained entirely in $S$.
\end{defn}

\begin{defn}
    The \textbf{convex hull} $\mathcal{CH}(S)$ of $S$ is the smallest convex set containing $S$.
    Specifically, it is the intersection of all convex sets containing $S$.
\end{defn}

\begin{itemize}
    \item Intuitively, if we have a set of points $P$ in the plane, we can get its convex hull by
    stretching a rubber band around all of the points, then letting go. So the 
    convex hull is the (unique) polygon with points from $P$ that contains all points in $P$.
    \item We can begin writing an algorithm to compute the convex hull of such a set. Let's represent a polygon by
    listing its vertices in clockwise order. Observe that given two points $p$, $q$ of $P$, the segment
    $\overline{pq}$ is an edge of $\mathcal{CH}(S)$ iff all other points of P lie to its right.
    This leads to the following algorithm:
\end{itemize}

\begin{algorithm}
    \SetKwData{Valid}{valid}\SetKwData{Edges}{E}
    \SetKwInOut{Input}{input}\SetKwInOut{Output}{output}

    \Input{A set $P$ of points in the plane.}
    \Output{A list $\mathcal{L}$ of the vertices of $\mathcal{CH}(S)$ in clockwise order.}
    \BlankLine
    \Edges$\gets \emptyset$.\\
    \For{\normalfont all ordered pairs $(p,q) \in P \times P$ with $p \neq q$}{
        \Valid$\gets \textbf{true}$\\
        \For{\normalfont all points $r \in P$ not equal to $p$ or $q$}{
            \If{\normalfont $r$ is to the left of the directed line from $p$ to $q$}{
                \Valid$\gets \textbf{false}$.
            }
        }
        \If{\Valid}{\normalfont Add the directed edge $\overrightarrow{pq}$ to \Edges .}
    }
    Construct $\mathcal{L}$ by sorting the elements of \Edges in clockwise order.\\
    \Return{$\mathcal{L}$}

    \caption{SlowConvexHull$(P)$}
\end{algorithm}

\begin{itemize}
    \item This algorithm has some issues, though. First is that it doesn't handle three points on one line.
    To fix this, we should require that $r$ be strictly to the right of $\overrightarrow{pq}$.
    \item Second, floating point computations can cause inaccuracies in the computation of the convex hull.
    For this algorithm, it may cause the result not to be a closed polygon at all!
    \item Lastly, it is slow. We check $n^2-n$ pairs of points, and $n-2$ points for each of those, making it
    run in $O(n^3)$ time.
\newpage
    \item The following algorithm rectifies many of the issues in the previous one.
    \item It works by sorting the points by $x$-coordinate (then by $y$-coordinate, if an
    $x$-coordinate is repeated), then adding the points in this order and updating the solution after each
    addition.
\end{itemize}

\begin{algorithm}
    \SetKwData{Valid}{valid}\SetKwData{Edges}{E}\SetKwData{i}{i}
    \SetKwInOut{Input}{input}\SetKwInOut{Output}{output}

    \Input{A set $P$ of points in the plane.}
    \Output{A list $\mathcal{L}$ of the vertices of $\mathcal{CH}(S)$ in clockwise order.}
    \BlankLine
    Sort $P$ by $x$- (then $y$-) coordinate, giving us a sequence $p_1, \dots, p_n$.\\
    \normalfont Add $p_1$ then $p_2$ to $\mathcal{L}_{upper}$.\\
    \For{\i $\gets 3$ \textbf{to} $n$}{
        Append $p_i$ to $\mathcal{L}_{upper}$.\\
        \While{\normalfont $\mathcal{L}_{upper}$ contains more than two points and the last three points in
                $\mathcal{L}_{upper}$ do not make a right turn}{
            Delete the middle of the last three points in $\mathcal{L}_{upper}$.
        }
    }
    Similarly compute the lower hull $\mathcal{L}_{lower}$.\\
    Delete the first and last points in $\mathcal{L}_{lower}$ to avoid duplication.\\
    Append $\mathcal{L}_{lower}$ to $\mathcal{L}_{upper}$.\\
    \Return{$\mathcal{L}$}

    \caption{ConvexHull$(P)$}
\end{algorithm}

\begin{itemize}
    \item This rectifies most of the problems with the last algorithm.
    \item Floating-point arithmetic may still cause errors, but not as badly as the last algorithm, which
    could have output something that wasn't even a closed polygon.
    \item The runtime for this is also dominated by the sorting, which can be done in $O(n\log n)$ time.
\end{itemize}

\section{Algorithm Strategies}
Development of a geometric algorithm often goes through three phases:
\begin{itemize}
    \item First, develop an algorithm ignoring degenerate cases that clutter our understanding of the problem.
    \item Second, adjust the algorithm to deal with degenerate cases, ideally by handling special cases with the general case.
    \item Third, actual implementation, where we must implement or find libraries to handle primitive operations like testing
    whether a point is to the left or right of a line, as well as consider the consequences of floating-point arithmetic.
\end{itemize}

\end{document}